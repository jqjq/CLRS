%% Generated by Sphinx.
\def\sphinxdocclass{report}
\documentclass[a4paper,10pt,english]{sphinxmanual}
\ifdefined\pdfpxdimen
   \let\sphinxpxdimen\pdfpxdimen\else\newdimen\sphinxpxdimen
\fi \sphinxpxdimen=.75bp\relax
%% turn off hyperref patch of \index as sphinx.xdy xindy module takes care of
%% suitable \hyperpage mark-up, working around hyperref-xindy incompatibility
\PassOptionsToPackage{hyperindex=false}{hyperref}
%% memoir class requires extra handling
\makeatletter\@ifclassloaded{memoir}
{\ifdefined\memhyperindexfalse\memhyperindexfalse\fi}{}\makeatother

\PassOptionsToPackage{warn}{textcomp}

\catcode`^^^^00a0\active\protected\def^^^^00a0{\leavevmode\nobreak\ }
\usepackage{cmap}
\usepackage{fontspec}
\defaultfontfeatures[\rmfamily,\sffamily,\ttfamily]{}
\usepackage{amsmath,amssymb,amstext}
\usepackage{polyglossia}
\setmainlanguage{english}



\setmainfont{FreeSerif}[
  Extension      = .otf,
  UprightFont    = *,
  ItalicFont     = *Italic,
  BoldFont       = *Bold,
  BoldItalicFont = *BoldItalic
]
\setsansfont{FreeSans}[
  Extension      = .otf,
  UprightFont    = *,
  ItalicFont     = *Oblique,
  BoldFont       = *Bold,
  BoldItalicFont = *BoldOblique,
]
\setmonofont{FreeMono}[
  Extension      = .otf,
  UprightFont    = *,
  ItalicFont     = *Oblique,
  BoldFont       = *Bold,
  BoldItalicFont = *BoldOblique,
]


\usepackage[Bjarne]{fncychap}
\usepackage[,numfigreset=0,mathnumfig]{sphinx}

\fvset{fontsize=\small}
\usepackage{geometry}


% Include hyperref last.
\usepackage{hyperref}
% Fix anchor placement for figures with captions.
\usepackage{hypcap}% it must be loaded after hyperref.
% Set up styles of URL: it should be placed after hyperref.
\urlstyle{same}


\usepackage{sphinxmessages}


\usepackage{preamble}

\title{Introduction to Algorithms}
\date{Apr 01, 2021}
\release{3.0}
\author{Thomas H. Cormen\and Charles E. Leiserson\and Ronald L. Rivest\and Clifford Stein}
\newcommand{\sphinxlogo}{\vbox{}}
\renewcommand{\releasename}{Release}
\makeindex
\begin{document}

\pagestyle{empty}
\sphinxmaketitle
\pagestyle{plain}
\sphinxtableofcontents
\pagestyle{normal}
\phantomsection\label{\detokenize{index::doc}}



\part{Selected Topics}
\label{\detokenize{part7/index:selected-topics}}\label{\detokenize{part7/index:part7}}\label{\detokenize{part7/index::doc}}

\chapter{String Matching}
\label{\detokenize{part7/ch32/index:string-matching}}\label{\detokenize{part7/ch32/index:ch32}}\label{\detokenize{part7/ch32/index::doc}}
\sphinxAtStartPar
Text\sphinxhyphen{}editing programs frequently need to find all occurrences of a pattern in
the text. Typically, the text is a document being edited, and the pattern
searched for is a particular word supplied by the user. Efficient algorithms
for this problem—called “string matching”—can greatly aid the responsiveness of
the text\sphinxhyphen{}editing program. Among their many other applications, string\sphinxhyphen{}matching
algorithms search for particular patterns in DNA sequences. Internet search
engines also use them to find Web pages relevant to queries.

\sphinxAtStartPar
We formalize the string\sphinxhyphen{}matching problem as follows. We assume that the text is
an array \(T[1 \twodots n]\) of length \(n\) and that the pattern is an
array \(P[1 \twodots m]\) of length \(m \le n\). We further assume that
the elements of \(P\) and \(T\) are characters drawn from a finite
alphabet \(\Sigma\). For example, we may have
\(\Sigma = \{ \mathtt{0} , \mathtt{1}\}\) or
\(\Sigma = \{ \mathtt{a,b,\ldots,z} \}\) \(\Sigma = \{ 1,2,3 \}\). The
character arrays \(P\) and \(T\) are often called \DUrole{strongemph}{strings}
of characters.



\renewcommand{\indexname}{Index}
\printindex
\end{document}